\documentclass{amsart}

%\usepackage{etoolbox}
%\makeatletter
%\let\ams@starttoc\@starttoc
%\makeatother
%\makeatletter
%\let\@starttoc\ams@starttoc
%\patchcmd{\@starttoc}{\makeatletter}{\makeatletter\parskip\z@}{}{}
%\makeatother

%\usepackage[parfill]{parskip}

\usepackage[colorlinks=true,linkcolor=blue,citecolor=blue,urlcolor=blue]{hyperref}
\usepackage{bookmark}
\usepackage{amsthm,thmtools,amssymb,amsmath,amscd}

\usepackage{fancyhdr}
\usepackage{esint}

\usepackage{enumerate}

\usepackage{pictexwd,dcpic}

\usepackage{graphicx}

\swapnumbers
\declaretheorem[name=Theorem,numberwithin=section]{thm}
\declaretheorem[name=Remark,style=remark,sibling=thm]{rem}
\declaretheorem[name=Lemma,sibling=thm]{lemma}
\declaretheorem[name=Proposition,sibling=thm]{prop}
\declaretheorem[name=Definition,style=definition,sibling=thm]{defn}
\declaretheorem[name=Corollary,sibling=thm]{cor}
\declaretheorem[name=Assumption,style=remark,sibling=thm]{ass}
\declaretheorem[name=Example,style=remark,sibling=thm]{example}


\numberwithin{equation}{section}

\usepackage{cleveref}
\crefname{lemma}{Lemma}{Lemmata}
\crefname{prop}{Proposition}{Propositions}
\crefname{thm}{Theorem}{Theorems}
\crefname{cor}{Corollary}{Corollaries}
\crefname{defn}{Definition}{Definitions}
\crefname{example}{Example}{Examples}
\crefname{rem}{Remark}{Remarks}
\crefname{ass}{Assumption}{Assumptions}
\crefname{not}{Notation}{Notation}

%Symbols
\renewcommand{\~}{\tilde}
\renewcommand{\-}{\bar}
\newcommand{\bs}{\backslash}
\newcommand{\cn}{\colon}
\newcommand{\sub}{\subset}

\newcommand{\N}{\mathbb{N}}
\newcommand{\R}{\mathbb{R}}
\newcommand{\Z}{\mathbb{Z}}
\renewcommand{\S}{\mathbb{S}}
\renewcommand{\H}{\mathbb{H}}
\newcommand{\C}{\mathbb{C}}
\newcommand{\K}{\mathbb{K}}
\newcommand{\Di}{\mathbb{D}}
\newcommand{\B}{\mathbb{B}}
\newcommand{\8}{\infty}

%Greek letters
\renewcommand{\a}{\alpha}
\renewcommand{\b}{\beta}
\newcommand{\g}{\gamma}
\renewcommand{\d}{\delta}
\newcommand{\e}{\epsilon}
\renewcommand{\k}{\kappa}
\renewcommand{\l}{\lambda}
\renewcommand{\o}{\omega}
\renewcommand{\t}{\theta}
\newcommand{\s}{\sigma}
\newcommand{\p}{\varphi}
\newcommand{\z}{\zeta}
\newcommand{\vt}{\vartheta}
\renewcommand{\O}{\Omega}
\newcommand{\D}{\Delta}
\newcommand{\G}{\Gamma}
\newcommand{\T}{\Theta}
\renewcommand{\L}{\Lambda}

%Mathcal Letters
\newcommand{\cL}{\mathcal{L}}
\newcommand{\cT}{\mathcal{T}}
\newcommand{\cA}{\mathcal{A}}
\newcommand{\cW}{\mathcal{W}}

%Mathematical operators
\newcommand{\INT}{\int_{\O}}
\newcommand{\DINT}{\int_{\d\O}}
\newcommand{\Int}{\int_{-\infty}^{\infty}}
\newcommand{\del}{\partial}

\newcommand{\inpr}[2]{\left\langle #1,#2 \right\rangle}
\newcommand{\fr}[2]{\frac{#1}{#2}}
\newcommand{\x}{\times}
\DeclareMathOperator{\Tr}{Tr}

\DeclareMathOperator{\dive}{div}
\DeclareMathOperator{\id}{id}
\DeclareMathOperator{\pr}{pr}
\DeclareMathOperator{\Diff}{Diff}
\DeclareMathOperator{\supp}{supp}
\DeclareMathOperator{\graph}{graph}
\DeclareMathOperator{\osc}{osc}
\DeclareMathOperator{\const}{const}
\DeclareMathOperator{\dist}{dist}
\DeclareMathOperator{\loc}{loc}
\DeclareMathOperator{\grad}{grad}
\DeclareMathOperator{\ric}{Ric}
\DeclareMathOperator{\Rm}{Rm}
\DeclareMathOperator{\weingarten}{\mathcal{W}}
\DeclareMathOperator{\inj}{inj}

%Environments
\newcommand{\Theo}[3]{\begin{#1}\label{#2} #3 \end{#1}}
\newcommand{\pf}[1]{\begin{proof} #1 \end{proof}}
\newcommand{\eq}[1]{\begin{equation}\begin{alignedat}{2} #1 \end{alignedat}\end{equation}}
\newcommand{\IntEq}[4]{#1&#2#3	 &\quad &\text{in}~#4,}
\newcommand{\BEq}[4]{#1&#2#3	 &\quad &\text{on}~#4}
\newcommand{\br}[1]{\left(#1\right)}



%Logical symbols
\newcommand{\Ra}{\Rightarrow}
\newcommand{\ra}{\rightarrow}
\newcommand{\hra}{\hookrightarrow}
\newcommand{\mt}{\mapsto}

% Aleksandrov Reflection Macros
\DeclareMathOperator{\reflectionvector}{V}
\DeclareMathOperator{\reflectionangle}{\delta}
\newcommand{\reflectionplane}[1][\reflectionvector]{\ensuremath{P_{#1}}}
\newcommand{\reflectionmap}[1][\reflectionvector]{\ensuremath{R_{#1}}}
\newcommand{\reflectionset}[2][\reflectionvector]{\ensuremath{{#2}_{#1}}}
\newcommand{\reflectionhalfspace}[1][\reflectionvector]{\ensuremath{\reflectionset[{#1}]{H}}}
\DeclareMathOperator{\vertvec}{e}
\DeclareMathOperator{\origin}{O}
\DeclareMathOperator{\radialprojection}{\pi}
\DeclareMathOperator{\height}{h}
\DeclareMathOperator{\equator}{E}
\newcommand{\ip}[2]{\ensuremath{\langle{#1},{#2}\rangle}}
\DeclareMathOperator{\intersect}{\cap}
\DeclareMathOperator{\union}{\cup}
\DeclareMathOperator{\nor}{\nu}
\DeclareMathOperator{\basepoint}{p_0}
\DeclareMathOperator{\radialdistance}{r}

%Fonts
\newcommand{\mc}{\mathcal}
\renewcommand{\it}{\textit}
\newcommand{\mrm}{\mathrm}

%Spacing
\newcommand{\hp}{\hphantom}


%\parindent 0 pt

\protected\def\ignorethis#1\endignorethis{}
\let\endignorethis\relax
\def\TOCstop{\addtocontents{toc}{\ignorethis}}
\def\TOCstart{\addtocontents{toc}{\endignorethis}}

\usepackage[bibstyle=alphabetic,citestyle=alphabetic,backend=bibtex]{biblatex}
\bibliography{Bibliography}

\begin{document}

\title[Ancient Solutions]
 {Ancient Solutions in Riemannian Backgrounds}

\curraddr{}
\email{}
\date{\today}

\dedicatory{}
\subjclass[2010]{}
\keywords{}

\begin{abstract}
\end{abstract}

\maketitle

\section{Introduction}
\label{sec:intro}

Let \(\bar{M}^{n+1}, \bar{g}\) be a compact Riemannian manifold with \(\bar{\ric} > 0\). A closed, embedded, ancient solution to the Mean Curvature Flow (MCF) is given by a smooth, one-parameter family of embeddings,
\[
F : M \times (-\infty, 0) \to \bar{M}
\]
with \(M\) a closed manifold, and satisfying
\begin{equation}
\label{eq:mcf}
F_{\ast} \partial_t = \vec{H} = -H\nu
\end{equation}
where \(\vec{H}\) denotes the Mean Curvature Vector Field, \(\nu\) is a choice of smooth, unit, normal vector field and \(H\) is the Mean Curvature with respect to \(\nu\): \(H = \Tr \weingarten\) with \(\weingarten (X) = -F_{\ast}^{-1} \bar{\nabla}_{F_{\ast} X} \nu\).

Let us write \(M_t = F_t(M)\), \(F_t (\cdot) = F(\cdot, t)\) and \(g = g_t = F^{\ast} \bar{g}\). We say that \(M_t\) is convex if for a choice of normal \(\nu\) we have \(A_t>0\) where
\[
A_t (X, Y) = g_t (\bar{\nabla}_{F_{\ast} X} F_{\ast} Y, \nu) = g_t(\weingarten(X), Y)
\]
is the second fundamental form with respect to \(\nu\) and \(\bar{\nabla}\) denotes the Levi-Civita connection for \(\bar{g}\) and \(A_t > 0\) means that \(A_t(X, X) > 0\) for all non-zero tangent vectors \(X\). We say that \(M_t\) is mean convex if only \(H > 0\) which is clearly implied by convexity, but not conversely.

\section{Linearisation and Stability}
\label{sec:linearising_linearisation}

In this section we consider the linearisation of the flow \eqref{eq:mcf} of graphs \(u : M_0 \to \R\) over a hypersurface \(M_0\). Arguing as in \cite[Lemmas 3.1, 3,2, 3.5]{Harltey:/2016} we obtain the linearisation is the Schr\"odinger operator,
\begin{equation}
\label{eq:linearisation}
-L u = \Delta u + \left(\overline{\ric}(\nu_0) + |\mathcal{W}|^2\right) u
\end{equation}
where derivatives and norms are computed with respect to the metric \(g_0 = X_0^{\ast} \bar{g}\) induced on \(M_0 = X_0(M)\), and Levi-Civita connection \(\nabla_0\), \(\nu_0\) denotes a smooth choice of unit normal along \(M_0\), and \(\bar{\ric}\) denotes the Ricci curvature of the ambient metric \(\bar{g}\).

For convenience, let us set
\[
V = -\left(\overline{\ric}_{\dot{F}}(\nu_0, \nu_0) + |\mathcal{W}|_{\dot{F}}^2\right)
\]
for the potential so that,
\[
L u = -\Delta u + V u.
\]

Integrating by parts over \(M_0\), we have the symmetric bilinear form,
\[
B(u, v) = \int_{M_0} v Lu = \int_{M_0} \inpr{\nabla u}{\nabla v} + V uv 
\]
and associated index form,
\[
I(u) = \int_{M_0} |\nabla u|^2 + V u^2.
\]

A particularly important feature is that $L u = 0$ is the Euler-Lagrange equation of the energy \(\tfrac{1}{2} I\) and so \(L\) is $L^2$ self-adjoint. Thus the spectrum of \(L\) is discrete, countable, and accumulating only at \(\infty\). Moreover \(L\) enjoys a variational characterisation of eigenvalues. In particular, the first eigenvalue,
\[
\lambda_1 = \inf \left\{I(u) : \int_{M_0} u = 1\right\}
\]
is simple with corresponding eigenfunction having a sign, which we take to be positive: \(Lu = \lambda_1 u \Rightarrow u > 0\). We write \(\lambda_1 < \lambda_2 \leq \lambda_3 \leq \cdots\) for the eigenvalues noting that \(\lambda_1 < 0\) and \(\sigma = \sigma(L)\) for the spectrum.

The index form is of course, precisely the index form for the area functional of minimal surfaces \(M_0\). In the particular case \(V < 0\) on \(M_0\) (for example if \(\ric > 0\)), by choosing \(u \equiv u_0\) is constant, we obtain the existence of a function \(u\) with \(I(u) < 0\). Equivalently, \(M_0\) has positive Morse index (there exists a negative eigenvalue \(I(u) = \lambda u\) with \(\lambda < 0\) for some \(u\)) if and only if \(M_0\) is unstable. In the case of positive Ricci, this is the well known result that all minimal surfaces are unstable.

The instability of minimal surfaces \(M_0\) implies the existence of ancient solutions limiting backwards to \(M_0\). Note that \(M_0\) is a fixed point of the flow \eqref{eq:mcf}.

\begin{thm}
The flow \eqref{eq:mcf} admits a non-static ancient solution \(M_t\) with \(M_t \to M_{-\infty}\) exponentially fast (in some appropriate topology, at least something like \(h^{1+\alpha}\) - just anything better than \(C^1\) probably).
\end{thm}

\begin{proof}
\cite[Theorem 9.1.3]{lunardi2012analytic} asserts that if \(\sup \{\text{Re}\lambda : \lambda \in \sigma(L)\} < 0\), and \(\sup\{\text{Re} \lambda : \lambda \in \sigma(L), \text{Re} \lambda < 0\} < 0\), then there exists non-static, ancient solutions with \(M_t \to M_{-\infty}\). Note that \cite{lunardi2012analytic} uses the convention \(L = \Delta - V\) and so the inequalities there are reversed. Since the eigenvalues of \(L\) are discrete, \(0\) is not an accumulation point, and \(\lambda_1 < 0\), these requirements are satisfied.
\end{proof}

\[
\sigma_{\pm} (L) = \{\lambda \in \sigma(L) : \pm \text{Re}(\lambda) > 0\}
\]
and
\begin{align*}
\omega_+ &= \inf \{\text{Re} \lambda : \lambda \in \sigma_+\} \\
- \omega_- &= \sup \{\text{Re} \lambda : \lambda \in \sigma_-\}.
\end{align*}

\printbibliography

\end{document}
