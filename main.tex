\documentclass{amsart}

%\usepackage{etoolbox}
%\makeatletter
%\let\ams@starttoc\@starttoc
%\makeatother
%\makeatletter
%\let\@starttoc\ams@starttoc
%\patchcmd{\@starttoc}{\makeatletter}{\makeatletter\parskip\z@}{}{}
%\makeatother

%\usepackage[parfill]{parskip}

\usepackage[colorlinks=true,linkcolor=blue,citecolor=blue,urlcolor=blue]{hyperref}
\usepackage{bookmark}
\usepackage{amsthm,thmtools,amssymb,amsmath,amscd}

\usepackage{fancyhdr}
\usepackage{esint}

\usepackage{enumerate}

\usepackage{pictexwd,dcpic}

\usepackage{graphicx}

\swapnumbers
\declaretheorem[name=Theorem,numberwithin=section]{thm}
\declaretheorem[name=Remark,style=remark,sibling=thm]{rem}
\declaretheorem[name=Lemma,sibling=thm]{lemma}
\declaretheorem[name=Proposition,sibling=thm]{prop}
\declaretheorem[name=Definition,style=definition,sibling=thm]{defn}
\declaretheorem[name=Corollary,sibling=thm]{cor}
\declaretheorem[name=Assumption,style=remark,sibling=thm]{ass}
\declaretheorem[name=Example,style=remark,sibling=thm]{example}


\numberwithin{equation}{section}

\usepackage{cleveref}
\crefname{lemma}{Lemma}{Lemmata}
\crefname{prop}{Proposition}{Propositions}
\crefname{thm}{Theorem}{Theorems}
\crefname{cor}{Corollary}{Corollaries}
\crefname{defn}{Definition}{Definitions}
\crefname{example}{Example}{Examples}
\crefname{rem}{Remark}{Remarks}
\crefname{ass}{Assumption}{Assumptions}
\crefname{not}{Notation}{Notation}

%Symbols
\renewcommand{\~}{\tilde}
\renewcommand{\-}{\bar}
\newcommand{\bs}{\backslash}
\newcommand{\cn}{\colon}
\newcommand{\sub}{\subset}

\newcommand{\N}{\mathbb{N}}
\newcommand{\R}{\mathbb{R}}
\newcommand{\Z}{\mathbb{Z}}
\renewcommand{\S}{\mathbb{S}}
\renewcommand{\H}{\mathbb{H}}
\newcommand{\C}{\mathbb{C}}
\newcommand{\K}{\mathbb{K}}
\newcommand{\Di}{\mathbb{D}}
\newcommand{\B}{\mathbb{B}}
\newcommand{\8}{\infty}

%Greek letters
\renewcommand{\a}{\alpha}
\renewcommand{\b}{\beta}
\newcommand{\g}{\gamma}
\renewcommand{\d}{\delta}
\newcommand{\e}{\epsilon}
\renewcommand{\k}{\kappa}
\renewcommand{\l}{\lambda}
\renewcommand{\o}{\omega}
\renewcommand{\t}{\theta}
\newcommand{\s}{\sigma}
\newcommand{\p}{\varphi}
\newcommand{\z}{\zeta}
\newcommand{\vt}{\vartheta}
\renewcommand{\O}{\Omega}
\newcommand{\D}{\Delta}
\newcommand{\G}{\Gamma}
\newcommand{\T}{\Theta}
\renewcommand{\L}{\Lambda}

%Mathcal Letters
\newcommand{\cL}{\mathcal{L}}
\newcommand{\cT}{\mathcal{T}}
\newcommand{\cA}{\mathcal{A}}
\newcommand{\cW}{\mathcal{W}}

%Mathematical operators
\newcommand{\INT}{\int_{\O}}
\newcommand{\DINT}{\int_{\d\O}}
\newcommand{\Int}{\int_{-\infty}^{\infty}}
\newcommand{\del}{\partial}

\newcommand{\inpr}[2]{\left\langle #1,#2 \right\rangle}
\newcommand{\fr}[2]{\frac{#1}{#2}}
\newcommand{\x}{\times}
\DeclareMathOperator{\Tr}{Tr}

\DeclareMathOperator{\dive}{div}
\DeclareMathOperator{\id}{id}
\DeclareMathOperator{\pr}{pr}
\DeclareMathOperator{\Diff}{Diff}
\DeclareMathOperator{\supp}{supp}
\DeclareMathOperator{\graph}{graph}
\DeclareMathOperator{\osc}{osc}
\DeclareMathOperator{\const}{const}
\DeclareMathOperator{\dist}{dist}
\DeclareMathOperator{\loc}{loc}
\DeclareMathOperator{\grad}{grad}
\DeclareMathOperator{\ric}{Ric}
\DeclareMathOperator{\Rm}{Rm}
\DeclareMathOperator{\weingarten}{\mathcal{W}}
\DeclareMathOperator{\inj}{inj}

%Environments
\newcommand{\Theo}[3]{\begin{#1}\label{#2} #3 \end{#1}}
\newcommand{\pf}[1]{\begin{proof} #1 \end{proof}}
\newcommand{\eq}[1]{\begin{equation}\begin{alignedat}{2} #1 \end{alignedat}\end{equation}}
\newcommand{\IntEq}[4]{#1&#2#3	 &\quad &\text{in}~#4,}
\newcommand{\BEq}[4]{#1&#2#3	 &\quad &\text{on}~#4}
\newcommand{\br}[1]{\left(#1\right)}



%Logical symbols
\newcommand{\Ra}{\Rightarrow}
\newcommand{\ra}{\rightarrow}
\newcommand{\hra}{\hookrightarrow}
\newcommand{\mt}{\mapsto}

% Aleksandrov Reflection Macros
\DeclareMathOperator{\reflectionvector}{V}
\DeclareMathOperator{\reflectionangle}{\delta}
\newcommand{\reflectionplane}[1][\reflectionvector]{\ensuremath{P_{#1}}}
\newcommand{\reflectionmap}[1][\reflectionvector]{\ensuremath{R_{#1}}}
\newcommand{\reflectionset}[2][\reflectionvector]{\ensuremath{{#2}_{#1}}}
\newcommand{\reflectionhalfspace}[1][\reflectionvector]{\ensuremath{\reflectionset[{#1}]{H}}}
\DeclareMathOperator{\vertvec}{e}
\DeclareMathOperator{\origin}{O}
\DeclareMathOperator{\radialprojection}{\pi}
\DeclareMathOperator{\height}{h}
\DeclareMathOperator{\equator}{E}
\newcommand{\ip}[2]{\ensuremath{\langle{#1},{#2}\rangle}}
\DeclareMathOperator{\intersect}{\cap}
\DeclareMathOperator{\union}{\cup}
\DeclareMathOperator{\nor}{\nu}
\DeclareMathOperator{\basepoint}{p_0}
\DeclareMathOperator{\radialdistance}{r}

%Fonts
\newcommand{\mc}{\mathcal}
\renewcommand{\it}{\textit}
\newcommand{\mrm}{\mathrm}

%Spacing
\newcommand{\hp}{\hphantom}


%\parindent 0 pt

\protected\def\ignorethis#1\endignorethis{}
\let\endignorethis\relax
\def\TOCstop{\addtocontents{toc}{\ignorethis}}
\def\TOCstart{\addtocontents{toc}{\endignorethis}}

\usepackage[bibstyle=alphabetic,citestyle=alphabetic,backend=bibtex]{biblatex}
\bibliography{Bibliography}

\begin{document}

\title[Ancient Solutions]
 {Ancient Solutions in Riemannian Backgrounds}

\curraddr{}
\email{}
\date{\today}

\dedicatory{}
\subjclass[2010]{}
\keywords{}

\begin{abstract}
\end{abstract}

\maketitle


\section{Notation and Conventions}
\label{sec:notation}

The speed function, \(F: \Gamma(T^{\ast}M \otimes TM) \to C^{\infty}(M)\) may be thought of as a function \(F : \Gamma(T^{\ast} M \otimes T^{\ast}M) \times \Gamma(T^{\ast} \odot T^{\ast}M)_+ \to C^{\infty}(M)\) via
\[
F(h, g) = F(\Tr_g h)
\]
where \(\Gamma(T^{\ast} \odot T^{\ast}M)_+\) denotes the fibre bundle of positive-definite, symmetric bilinear forms. 

Writing
\[
\dot{F}\left|_A\right. (B) = \left.\frac{d}{ds} \right|_{s=0} F(A + s B),
\]
we obtain a positive definite (from the parabolicity condition), symmetric bi-linear form on \(T^{\ast}M\),
\[
B_{\dot{F}} (\alpha, \beta) = \dot{F} ( \alpha \otimes \beta).
\]

With respect to a frame $\{E_i\}$, we may write
\[
\dot{F} (B) = \dot{F}^{ij} B_{ij}.
\]
where $B_{ij} = B(E_i, E_j)$ and $\dot{F}^{ij} = \dot{F} (dx^i \otimes dx^j)$. Letting $\theta^i$ denote the dual frame ($\theta^i(E_j) = \delta^i_j$) and writing $\alpha = \alpha_i \theta^i$, $\beta = \beta_j \theta^j$ we have,
\[
B_{\dot{F}} (\alpha, \beta) = \dot{F}^{ij} \alpha_i \beta_j.
\]

We can use \(B_{\dot{F}}\) to define musical isomorphisms and so raise and lower indices. In particular, denote by 
\[
\ric_{\dot{F}} = \Tr_{B_{\dot{F}}} \Rm
\]
the trace of \(\Rm\) with respect to \(B_{\dot{F}}\), written with respect to a frame as
\[
(\ric_{\dot{F}})_{ij} = \dot{F}^{kl} \Rm_{kijl}.
\]
We also have associated norms
\[
|\alpha|_{\dot{F}}^2 = B_{\dot{F}}(\alpha, \alpha)
\]
which induces, in particular, a norm on endomorphisms \(T \in T^{\ast}M \otimes TM\)
\[
|T|_{\dot{F}}^2 = \dot{F}^{ij} g_{il} T^l_k T^k_j.
\]

Let us also define the operator
\[
\Box u = \Box_{\dot{F}} u = \dot{F}(\nabla^2 u) = \dot{F}^{ij} \nabla^2_{ij} u.
\]

\section{Linearisation and Stability}
\label{sec:linearising_linearisation}

The linearisation of the flow of graphs over a hypersurface $M_0$ is given in \cite[Lemma 3.5]{Harltey:/2016}. Compared with the setting there, in our situation there is no weight function: \(\Xi = 0\) and so the integral terms disappear. Moreover, we do not assume that \(M_0\) is totally umbilic. Let us rephrase the result of \cite[Lemma 3.5]{Harltey:/2016} in the notation of \Cref{sec:notation}.

Arguing as in \cite[Lemmas 3.1, 3,2, 3.5]{Harltey:/2016} we obtain the linearisation is the Schr\"odinger operator,
\begin{equation}
\label{eq:linearisation}
\tilde{L} u = \tilde{L}_{\dot{F}} u = \Box u + \left(\overline{\ric}_{\dot{F}}(\nu_0, \nu_0) + |\mathcal{W}|_{\dot{F}}^2\right) u
\end{equation}
where derivatives and norms are computed with respect to the metric \(g_0 = X_0^{\ast} \bar{g}\) induced on \(M_0 = X_0(M)\), and Levi-Civita connection \(\nabla_0\).

We work with the operator \(L = -\tilde{L}\) since \(-\Box\) is a positive operator (see \Cref{rem:mcflinearisation} below). For convenience, let us set
\[
V = -\left(\overline{\ric}_{\dot{F}}(\nu_0, \nu_0) + |\mathcal{W}|_{\dot{F}}^2\right)
\]
for the potential so that,
\[
L u = -\Box u + V u.
\]

\begin{rem}
\label{rem:mcflinearisation}
In the case of the Mean Curvature Flow, \(\Box = \Delta\), the Laplacian, \(\ric_{\dot{F}} = \ric\), and \(|\mathcal{W}|_{\dot{F}} = |\mathcal{W}|\). Thus we have
\[
L u = -\Delta u - (\ric(\nu_0, \nu_0) + |\mathcal{W}|^2) u.
\]
Integrating by parts over \(M_0\), we have the symmetric bilinear form,
\[
B(u, v) = \int_{M_0} v Lu = \int_{M_0} \inpr{\nabla u}{\nabla v} + V uv 
\]
and associated index form,
\[
I(u) = \int_{M_0} |\nabla u|^2 + V u^2.
\]
The index form is of course, precisely the index form for the area functional of minimal surfaces. In the particular case \(V < 0\) on \(M_0\) (for example if \(\ric > 0\)), by choosing \(u \equiv u_0\) is constant, we obtain the existence of a function \(u\) with \(I(u) < 0\). Equivalently, \(M_0\) has positive Morse index (there exists a negative eigenvalue \(I(u) = \lambda u\) with \(\lambda < 0\) for some $u$) if and only if \(M_0\) is unstable. In the case of positive Ricci, this is the well known result that all minimal surfaces are unstable.

A particularly important feature is that $\Delta$ arises in the Euler-Lagrange equation of the Dirichlet energy $\int_{M_0} |\nabla u|^2$ and so the operator $L = -\Delta + V$ is $L^2$ self adjoint and enjoys a variational characterisation of eigenvalues. In particular, the first eigenvalue,
\[
\lambda_1 = \inf \left\{\int_{M_0} u L u : \int_{M_0} u = 1\right\}
\]
is simple with corresponding eigenfuction having a sign \(u > 0\) (or \(u < 0\) depending on convention).
\end{rem}

We adopt the notation from \cite[Chapters 8, 9]{lunardi2012analytic} for notions relating to the spectrum \(\sigma = \sigma(L)\) of \(L\). Define,
\[
\sigma_{\pm} (L) = \{\lambda \in \sigma(L) : \pm \text{Re}(\lambda) > 0\}
\]
and
\begin{align*}
\omega_+ &= \inf \{\text{Re} \lambda : \lambda \in \sigma_+\} \\
- \omega_- &= \sup \{\text{Re} \lambda : \lambda \in \sigma_-\}.
\end{align*}

\printbibliography

\end{document}
