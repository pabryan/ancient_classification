\documentclass{amsart}

\input{StandardPaper2.tex}
\usepackage[bibstyle=alphabetic,citestyle=alphabetic,backend=bibtex]{biblatex}
\bibliography{Bibliography}

\begin{document}

\title[Ancient Solutions]
 {Ancient Solutions in Riemannian Backgrounds}

\curraddr{}
\email{}
\date{\today}

\dedicatory{}
\subjclass[2010]{}
\keywords{}

\begin{abstract}
\end{abstract}

\maketitle

\section{Introduction}
\label{sec:intro}

Let \(\bar{M}^{n+1}, \bar{g}\) be a compact Riemannian manifold with \(\bar{\ric} > 0\). A closed, embedded, ancient solution to the Mean Curvature Flow (MCF) is given by a smooth, one-parameter family of embeddings,
\[
F : M \times (-\infty, T) \to \bar{M}
\]
with \(M\) a closed manifold, and satisfying
\begin{equation}
\label{eq:mcf}
F_{\ast} \partial_t = \vec{H} = -H\nu
\end{equation}
where \(\vec{H}\) denotes the Mean Curvature Vector Field, \(\nu\) is a choice of smooth, unit, normal vector field and \(H\) is the Mean Curvature with respect to \(\nu\): \(H = \Tr \weingarten\) with \(\weingarten (X) = -F_{\ast}^{-1} \bar{\nabla}_{F_{\ast} X} \nu\). Note also that we allow \(T = \infty\) so that any closed, minimal surface is a (static), ancient solution.

Let us write \(M_t = F_t(M)\), \(F_t (\cdot) = F(\cdot, t)\) and \(g = g_t = F^{\ast} \bar{g}\). We say that \(M_t\) is convex if for a choice of normal \(\nu\) we have \(A_t>0\) where
\[
A_t (X, Y) = g_t (\bar{\nabla}_{F_{\ast} X} F_{\ast} Y, \nu) = g_t(\weingarten(X), Y)
\]
is the second fundamental form with respect to \(\nu\) and \(\bar{\nabla}\) denotes the Levi-Civita connection for \(\bar{g}\) and \(A_t > 0\) means that \(A_t(X, X) > 0\) for all non-zero tangent vectors \(X\). We say that \(M_t\) is mean-convex if only \(H > 0\) which is clearly implied by convexity, but not conversely.

An important tool we employ to study ancient solutions is the principle of linearised stablity. Let \(M_0 \subset \bar{M}\) be a closed, smooth, two-sided, minimal hypersurface. Then \(M_0\) is a fixed point for the flow \eqref{eq:mcf}. In this instance, we work only with smooth minimal hypersurfaces. By \cite{pitts:/1983}, the existence of smooth, closed minimal surfaces is assured up to dimenion six. Of course, smooth, closed minimal hypersurfaces may exist in higher dimensions also (e.g. equators in spheres) and our results also apply to such minimal hypersurfaces.

Let \(X = h^{2,\alpha}(M_0)\) be the little \holder{} space, which by definition is the completion of \(C^{\infty}(M_0)\) with respect to the \holder{} norm \(\|\cdot\|_{2,\alpha}\) in the \holder{} space \(C^{2,\alpha}(M)\). Such spaces enjoy a self-interpolation property crucial to our study \cite[Equation 19]{guenther2002stability}. We consider ancient solutions of \eqref{eq:mcf} that are normal graphs,
\[
F(x, t) = \exp(u(x, t) \nu_0(x)), \quad x \in M_0
\]
where \(\exp\) is the exponential map on \(\bar{M}\) and \(\nu_0\) is a smooth unit normal field along \(M_0\) and \(u \in C^1((-\infty, 0), X)\). We have that \(F\) is a solution of \eqref{eq:mcf} if and only if \(u\) solves,
\begin{equation}
\label{eq:graph_mcf}
\partial_t u = \sqrt{1 + |\grad u|^2} \div\left(\frac{\grad u}{\sqrt{1 + |\grad u|^2}}\right)
\end{equation}
where \(|\cdot|\) is induced by the metric along \(M_t\) pulled back to \(M_0\) and gradients are also taken with respect to this metric.

Note that \(u \equiv 0\) is just \(M_0\). A simple application of the principle of linearized stability recorded in \Cref{thm:ancient_existence} shows that there do exist such ancient normal graph solutions to \eqref{eq:mcf} converging exponential to \(M_0\) as \(t \to - \infty\). This is just a consequence of the fact that in positive Ricci curvature backgrounds, all minimal hypersurfaces are unstable. Moreover, in \Cref{thm:ancient_meanconvex_uniqueness} we show that amongst such ancient solutions, there exists a unique, mean-convex one. Such a solution provides a foliation of \(\bar{M}\) in a neighbourhood of \(M_0\).

The above discussion raises the question of whether every ancient solution must emanate from a closed minimal hypersurface. That is, if \(M_t\) is an ancient solution, does it converge to a minimal hypersurface as \(t \to -\infty\)? By making use of the Harnack inequality, in \Cref{sec:harnack_backwards}, we prove that any ancient, convex solution does indeed converge backwards to a minimal hypersurface. This minimal hypersurface is in fact totally geodesic and any convex, ancient solution is also totally geodeisc and in particular has constant mean curvature. Thus in particular, a totally geodesic, closed hypersurface \(M_0\) exists in \(\bar{M}\) if and only if a convex, ancient solution exists. Moreover, in the case of existence, a neighbourhood of \(M_0\) is foliated by totally geodesic hypersurfaces.

\section{Linearisation and Stability}
\label{sec:linearising_linearisation}

In this section we consider the linearisation of the flow \eqref{eq:graph_mcf} of normal graphs \(u \in X\) over a closed, minimal hypersurface \(M_0\). Arguing as in \cite[Lemmas 3.1, 3,2, 3.5]{Harltey:/2016} we obtain the linearisation is the Schr\"odinger operator,
\begin{equation}
\label{eq:linearisation}
-L u = \Delta u + \left(\overline{\ric}(\nu_0) + |\mathcal{W}|^2\right) u
\end{equation}
where derivatives and norms are computed with respect to the metric \(g_0 = F_0^{\ast} \bar{g}\) induced on \(M_0 = X_0(M)\), and Levi-Civita connection \(\nabla_0\), \(\nu_0\) denotes a smooth choice of unit normal along \(M_0\), and \(\bar{\ric}\) denotes the Ricci curvature of the ambient metric \(\bar{g}\).

For convenience, let us set
\[
V = -\left(\overline{\ric}_{\dot{F}}(\nu_0, \nu_0) + |\mathcal{W}|_{\dot{F}}^2\right)
\]
for the potential so that,
\[
L u = -\Delta u + V u.
\]

Integrating by parts over \(M_0\), we have the symmetric bilinear form,
\[
B(u, v) = \int_{M_0} v Lu = \int_{M_0} \inpr{\nabla u}{\nabla v} + V uv 
\]
and associated index form,
\[
I(u) = \int_{M_0} |\nabla u|^2 + V u^2.
\]

A particularly important feature is that $L u = 0$ is the Euler-Lagrange equation of the energy \(\tfrac{1}{2} I\) and so \(L\) is $L^2$ self-adjoint. Thus the spectrum of \(L\) is discrete, countable, and accumulating only at \(\infty\). Moreover \(L\) enjoys a variational characterisation of eigenvalues. In particular, the first eigenvalue,
\[
\lambda_1 = \inf \left\{I(u) : \int_{M_0} u = 1\right\}
\]
is simple with corresponding eigenfunction having a sign, which we take to be positive: \(Lu = \lambda_1 u \Rightarrow u > 0\). We write \(\lambda_1 < \lambda_2 \leq \lambda_3 \leq \cdots\) for the eigenvalues noting that \(\lambda_1 < 0\) and \(\sigma = \sigma(L)\) for the spectrum.

The index form is of course, precisely the index form for the area functional of minimal surfaces \(M_0\). In the particular case \(V < 0\) on \(M_0\) (for example if \(\ric > 0\)), by choosing \(u \equiv u_0\) is constant, we obtain the existence of a function \(u\) with \(I(u) < 0\). Equivalently, \(M_0\) has positive Morse index (there exists a negative eigenvalue \(I(u) = \lambda u\) with \(\lambda < 0\) for some \(u\)) if and only if \(M_0\) is unstable. In the case of positive Ricci, this is the well known result that all minimal surfaces are unstable.

The instability of minimal surfaces \(M_0\) implies the existence of ancient solutions limiting backwards to \(M_0\). Note that \(M_0\) is a fixed point of the flow \eqref{eq:mcf}.

\begin{thm}
\label{thm:ancient_existence}
The flow \eqref{eq:mcf} admits a non-static ancient solution \(M_t\) with \(M_t \to M_{-\infty}\) exponentially fast (in some appropriate topology, at least something like \(h^{1+\alpha}\) - just anything better than \(C^1\) probably).
\end{thm}

\begin{rem}
In the proof, note that we use \(L = -\Delta - V\), whereas \cite{lunardi2012analytic} uses the convention \(L = \Delta + V\). Thus the negative real part of the spectrum \(\{\lambda \in \sigma : \text{Re} \lambda < 0\}\) used here corresponds to the positive part of the specturm \(\{\lambda \in \sigma : \text{Re} \lambda > 0\}\) in \cite{lunardi2012analytic}.
\end{rem}

\begin{proof}
\cite[Theorem 9.1.3]{lunardi2012analytic} asserts that if \(\sup \{\text{Re}\lambda : \lambda \in \sigma(L)\} < 0\), and \(\sup\{\text{Re} \lambda : \lambda \in \sigma(L), \text{Re} \lambda < 0\} < 0\), then there exists non-static, ancient solutions with \(M_t \to M_{-\infty}\) exponentially fast in \(h^{2,\alpha}\). Since the eigenvalues of \(L\) are discrete, \(0\) is not an accumulation point and \(\lambda_1 < 0\), these requirements are satisfied.
\end{proof}

Actually, much more is known about solutions near \(M_0\). There exists a finite dimensional \emph{unstable-center manifold}, consisting of nearby ancient solutions (including static ones), which exponentially attracts all other nearby solutions. Let us recall the construction.

Write,
\[
\sigma^- = \{\lambda \in \sigma : \lambda \leq 0\}.
\]
Then \(\sigma^-\) is a non-empty spectral set consisting of a finite number of eigenvalues, \(\lambda_1 < \lambda_2 < \cdots \leq \lambda_k \leq 0\) with finite algebraic multiplicity. Let \(P: X \to X\) be the projection associated with the spectral set \(\sigma^-\), which in this case is the \(L^2\)-orthogonal projection
\[
P : X \to N^- = \{u \in X : L(u) = \lambda u \text{ for some } \lambda \in \sigma^-\}.
\]
As in \cite[Theorem 4.1]{Simonett:/1995} (see also \cite[Theorem 9.2.2]{lunardi2012analytic}), there exists an \(R_1 \geq 0\) and a Lipschitz continuous function, \(\Gamma : B_{R_1}(X) \subset P(X) \to (\operatorname{Id} - P)(X)\) such that
\[
\mathcal{M} = \{\Gamma(u) : u \in B_{R_1}(0)\}
\]
has the following properties:
\begin{enumerate}
\item For every \(u_0 \in \mathcal{M}\), there exists a unique ancient solution \(u\) such that \(\lim_{t\to-\infty} \|u\| = 0\) and \(u(0) = u_0\),
\item if \(u\) is an ancient solution such that \(\|u\| \leq R_1\), then \(u \in \mathcal{M}\),
\item for each \(k \in \N\), there exists \(R_k > 0\) such that \(\Gamma \in C^{k-1,1}\) on \(B_{R_k}(0)\).
\end{enumerate}

The graph \(\mathcal{M}\) is known as the \emph{center-unstable manifold}. It is invariant under the flow \eqref{eq:mcf} in the sense that if \(u_0 \in \mathcal{M}\) and \(u(t)\) is the solution of \eqref{eq:mcf} with \(u(0) = u_0\), then \(u(t) \in \mathcal{M}\) provided \(\|u\| \leq R_1\). Thus \(\mathcal{M}\) contains all ancient solutions emanating from \(M_0\); \(\Gamma(\{u : Lu = 0\})\) is comprised of static solutions, while \(\Gamma(\{u : Lu < 0\})\) is comprised of non-static solutions.

Since \(\lambda_1 < 0\), as noted in \Cref{thm:ancient_existence}, there exist non-static, ancient solutions emanating from \(M_0\). If \(\dim \{u : L u < 0\} > 1\), we won't have uniqueness of such ancient solutions. However, since the first eigenvalue is simple, we have the following uniqueness result.

\begin{thm}
\label{thm:ancient_meanconvex_uniqueness}
There exists a unique, mean-convex, ancient solution emanating from \(M_0\). In particular, if \(M_0\) is an index one minimal surface, then there is a unique, ancient solution emanating from \(M_0\) and this solution is mean-convex.
\end{thm}

\begin{proof}
Mean-Convexity corresponds to positivity of the corresponding eigenfunction which is unique since \(\lambda_1\) is simple, and orthogonality of eigenfunctions implies all other eigenfunctions must change sign.
\end{proof}

We also have that in certain situations, the symmetries of \(M_0\) are inherited by any ancient solution emanating from it. We have that for some \(\epsilon > 0\), the map
\[
\Phi (x, r) \in M_0 \times (-\epsilon, \epsilon) \mapsto \exp_x(r\nu)
\]
is a diffeomorphism onto an open neighbourhood of \(M_0\). Let,
\[
\Phi^{\ast} \bar{g} = g_r(x) + dr^2
\]
denote the induced metric on \(M_0 \times (-\epsilon, \epsilon)\) where \(g_r\) is a smooth, one-parameter family of metrics on \(M_0\). This expression for the right hand side follows from Gauss' lemma. We may also write \(g_r = \Phi_r^{\ast} \bar{g}\) where \(\Phi_r (\cdot) = \Phi(\cdot, r)\).

\begin{thm}
Let \(G\) be a Lie group acting on \(M_0\) such that \(g_r\) is invariant under \(G\) for \(r \in (-\epsilon, \epsilon)\). Then any ancient solution emanating from \(M_0\) is invariant under the action of \(G\) whilst it remains within \(\Phi(M_0 \times (-\epsilon, \epsilon))\).
\end{thm}

\begin{proof}
Write \((x, g) \in M_0 \times G \mapsto gx \in M_0\) for the action of \(G\) on \(M_0\) and for each \(g \in G\), define \(L_g : x \mapsto gx\). The assumption is that \(L_g^{\ast} g_r = g_r\), and we need to prove that \(u(L_g(x)) = u(x)\) for each \(x \in M_0\) and each \(g \in G\).

From the assumptions, we find that
\[
g_r(x) = e^{f(r)} g_0
\]
for some smooth function \(f(r)\).

Now it should be relatively straight forward.
\end{proof}

\begin{cor}
Suppose the assumptions as in the theorem hold, and in addition that \(G\) acts transitively on \(M_0\). Then any ancient emanating from \(M_0\) is by CMC hypersurfaces. In addition, if \(M_0\) is totally geodesic (umbilic), then any ancient solution \(M_t\) emanating from \(M_0\) is totally geodesic (resp. umbilic) while it remains in \(M_0 \times (-\epsilon, \epsilon)\).
\end{cor}

\begin{proof}

\end{proof}

\section{The Harnack Inequality and Backwards Convergence}
\label{sec:harnack_backwards}

\printbibliography

\end{document}
