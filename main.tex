\documentclass{amsart}

%\usepackage{etoolbox}
%\makeatletter
%\let\ams@starttoc\@starttoc
%\makeatother
%\makeatletter
%\let\@starttoc\ams@starttoc
%\patchcmd{\@starttoc}{\makeatletter}{\makeatletter\parskip\z@}{}{}
%\makeatother

%\usepackage[parfill]{parskip}

\usepackage[colorlinks=true,linkcolor=blue,citecolor=blue,urlcolor=blue]{hyperref}
\usepackage{bookmark}
\usepackage{amsthm,thmtools,amssymb,amsmath,amscd}

\usepackage{fancyhdr}
\usepackage{esint}

\usepackage{enumerate}

\usepackage{pictexwd,dcpic}

\usepackage{graphicx}

\swapnumbers
\declaretheorem[name=Theorem,numberwithin=section]{thm}
\declaretheorem[name=Remark,style=remark,sibling=thm]{rem}
\declaretheorem[name=Lemma,sibling=thm]{lemma}
\declaretheorem[name=Proposition,sibling=thm]{prop}
\declaretheorem[name=Definition,style=definition,sibling=thm]{defn}
\declaretheorem[name=Corollary,sibling=thm]{cor}
\declaretheorem[name=Assumption,style=remark,sibling=thm]{ass}
\declaretheorem[name=Example,style=remark,sibling=thm]{example}


\numberwithin{equation}{section}

\usepackage{cleveref}
\crefname{lemma}{Lemma}{Lemmata}
\crefname{prop}{Proposition}{Propositions}
\crefname{thm}{Theorem}{Theorems}
\crefname{cor}{Corollary}{Corollaries}
\crefname{defn}{Definition}{Definitions}
\crefname{example}{Example}{Examples}
\crefname{rem}{Remark}{Remarks}
\crefname{ass}{Assumption}{Assumptions}
\crefname{not}{Notation}{Notation}

%Symbols
\renewcommand{\~}{\tilde}
\renewcommand{\-}{\bar}
\newcommand{\bs}{\backslash}
\newcommand{\cn}{\colon}
\newcommand{\sub}{\subset}

\newcommand{\N}{\mathbb{N}}
\newcommand{\R}{\mathbb{R}}
\newcommand{\Z}{\mathbb{Z}}
\renewcommand{\S}{\mathbb{S}}
\renewcommand{\H}{\mathbb{H}}
\newcommand{\C}{\mathbb{C}}
\newcommand{\K}{\mathbb{K}}
\newcommand{\Di}{\mathbb{D}}
\newcommand{\B}{\mathbb{B}}
\newcommand{\8}{\infty}

%Greek letters
\renewcommand{\a}{\alpha}
\renewcommand{\b}{\beta}
\newcommand{\g}{\gamma}
\renewcommand{\d}{\delta}
\newcommand{\e}{\epsilon}
\renewcommand{\k}{\kappa}
\renewcommand{\l}{\lambda}
\renewcommand{\o}{\omega}
\renewcommand{\t}{\theta}
\newcommand{\s}{\sigma}
\newcommand{\p}{\varphi}
\newcommand{\z}{\zeta}
\newcommand{\vt}{\vartheta}
\renewcommand{\O}{\Omega}
\newcommand{\D}{\Delta}
\newcommand{\G}{\Gamma}
\newcommand{\T}{\Theta}
\renewcommand{\L}{\Lambda}

%Mathcal Letters
\newcommand{\cL}{\mathcal{L}}
\newcommand{\cT}{\mathcal{T}}
\newcommand{\cA}{\mathcal{A}}
\newcommand{\cW}{\mathcal{W}}

%Mathematical operators
\newcommand{\INT}{\int_{\O}}
\newcommand{\DINT}{\int_{\d\O}}
\newcommand{\Int}{\int_{-\infty}^{\infty}}
\newcommand{\del}{\partial}

\newcommand{\inpr}[2]{\left\langle #1,#2 \right\rangle}
\newcommand{\fr}[2]{\frac{#1}{#2}}
\newcommand{\x}{\times}
\DeclareMathOperator{\Tr}{Tr}

\DeclareMathOperator{\dive}{div}
\DeclareMathOperator{\id}{id}
\DeclareMathOperator{\pr}{pr}
\DeclareMathOperator{\Diff}{Diff}
\DeclareMathOperator{\supp}{supp}
\DeclareMathOperator{\graph}{graph}
\DeclareMathOperator{\osc}{osc}
\DeclareMathOperator{\const}{const}
\DeclareMathOperator{\dist}{dist}
\DeclareMathOperator{\loc}{loc}
\DeclareMathOperator{\grad}{grad}
\DeclareMathOperator{\ric}{Ric}
\DeclareMathOperator{\Rm}{Rm}
\DeclareMathOperator{\weingarten}{\mathcal{W}}
\DeclareMathOperator{\inj}{inj}

%Environments
\newcommand{\Theo}[3]{\begin{#1}\label{#2} #3 \end{#1}}
\newcommand{\pf}[1]{\begin{proof} #1 \end{proof}}
\newcommand{\eq}[1]{\begin{equation}\begin{alignedat}{2} #1 \end{alignedat}\end{equation}}
\newcommand{\IntEq}[4]{#1&#2#3	 &\quad &\text{in}~#4,}
\newcommand{\BEq}[4]{#1&#2#3	 &\quad &\text{on}~#4}
\newcommand{\br}[1]{\left(#1\right)}



%Logical symbols
\newcommand{\Ra}{\Rightarrow}
\newcommand{\ra}{\rightarrow}
\newcommand{\hra}{\hookrightarrow}
\newcommand{\mt}{\mapsto}

% Aleksandrov Reflection Macros
\DeclareMathOperator{\reflectionvector}{V}
\DeclareMathOperator{\reflectionangle}{\delta}
\newcommand{\reflectionplane}[1][\reflectionvector]{\ensuremath{P_{#1}}}
\newcommand{\reflectionmap}[1][\reflectionvector]{\ensuremath{R_{#1}}}
\newcommand{\reflectionset}[2][\reflectionvector]{\ensuremath{{#2}_{#1}}}
\newcommand{\reflectionhalfspace}[1][\reflectionvector]{\ensuremath{\reflectionset[{#1}]{H}}}
\DeclareMathOperator{\vertvec}{e}
\DeclareMathOperator{\origin}{O}
\DeclareMathOperator{\radialprojection}{\pi}
\DeclareMathOperator{\height}{h}
\DeclareMathOperator{\equator}{E}
\newcommand{\ip}[2]{\ensuremath{\langle{#1},{#2}\rangle}}
\DeclareMathOperator{\intersect}{\cap}
\DeclareMathOperator{\union}{\cup}
\DeclareMathOperator{\nor}{\nu}
\DeclareMathOperator{\basepoint}{p_0}
\DeclareMathOperator{\radialdistance}{r}

%Fonts
\newcommand{\mc}{\mathcal}
\renewcommand{\it}{\textit}
\newcommand{\mrm}{\mathrm}

%Spacing
\newcommand{\hp}{\hphantom}


%\parindent 0 pt

\protected\def\ignorethis#1\endignorethis{}
\let\endignorethis\relax
\def\TOCstop{\addtocontents{toc}{\ignorethis}}
\def\TOCstart{\addtocontents{toc}{\endignorethis}}

\usepackage[bibstyle=alphabetic,citestyle=alphabetic,backend=bibtex]{biblatex}
\bibliography{Bibliography}

\begin{document}

\title[Ancient Solutions]
 {Ancient Solutions in Riemannian Backgrounds}

\curraddr{}
\email{}
\date{\today}

\dedicatory{}
\subjclass[2010]{}
\keywords{}

\begin{abstract}
\end{abstract}

\maketitle

\section{Introduction}
\label{sec:intro}

Let \(\bar{M}^{n+1}, \bar{g}\) be a compact Riemannian manifold with \(\bar{\ric} > 0\). A closed, embedded, ancient solution to the Mean Curvature Flow (MCF) is given by a smooth, one-parameter family of embeddings,
\[
F : M \times (-\infty, T) \to \bar{M}
\]
with \(M\) a closed manifold, and satisfying
\begin{equation}
\label{eq:mcf}
F_{\ast} \partial_t = -f(\weingarten) \nu
\end{equation}
where \(\nu\) is a choice of smooth, unit, normal vector field, \(\weingarten (X) = -F_{\ast}^{-1} \bar{\nabla}_{F_{\ast} X} \nu\) is the Weingarten map \(\weingarten \in C^{\infty}(T^{\ast}M \otimes TM)\) and \(f\) is a so-called, \emph{curvature function}. Note also that we allow \(T = \infty\) so that in particular, any closed, minimal surface is a (static), ancient solution.

Let us write \(M_t = F_t(M)\), \(F_t (\cdot) = F(\cdot, t)\) and \(g = g_t = F^{\ast} \bar{g}\). We say that \(M_t\) is convex if for a choice of normal \(\nu\) we have \(A_t>0\) where
\[
A_t (X, Y) = g_t (\bar{\nabla}_{F_{\ast} X} F_{\ast} Y, \nu) = g_t(\weingarten(X), Y)
\]
is the second fundamental form with respect to \(\nu\) and \(\bar{\nabla}\) denotes the Levi-Civita connection for \(\bar{g}\) and \(A_t > 0\) means that \(A_t(X, X) > 0\) for all non-zero tangent vectors \(X\). We say that \(M_t\) is mean-convex if only \(H > 0\) which is clearly implied by convexity, but not conversely.

An important tool we employ to study ancient solutions is the principle of linearised stablity. Let \(M_0 \subset \bar{M}\) be a closed, smooth, two-sided, minimal hypersurface. Then \(M_0\) is a fixed point for the flow \eqref{eq:mcf}. In this instance, we work only with smooth minimal hypersurfaces. By \cite{pitts:/1983}, the existence of smooth, closed minimal surfaces is assured up to dimension six. Of course, smooth, closed minimal hypersurfaces may exist in higher dimensions also (e.g. equators in spheres) and our results also apply to such minimal hypersurfaces.

Let \(X = h^{2,\alpha}(M_0)\) be the little \holder{} space, which by definition is the completion of \(C^{\infty}(M_0)\) with respect to the \holder{} norm \(\|\cdot\|_{2,\alpha}\) in the \holder{} space \(C^{2,\alpha}(M)\). Such spaces enjoy a self-interpolation property crucial to our study \cite[Equation 19]{guenther2002stability}. We consider ancient solutions of \eqref{eq:mcf} that are normal graphs,
\[
F(x, t) = \exp(u(x, t) \nu_0(x)), \quad x \in M_0
\]
where \(\exp\) is the exponential map on \(\bar{M}\) and \(\nu_0\) is a smooth unit normal field along \(M_0\) and \(u \in C^1((-\infty, 0), X)\). We have that \(F\) is a solution of \eqref{eq:mcf} if and only if \(u\) solves,
\begin{equation}
\label{eq:graph_mcf}
\partial_t u = \sqrt{1 + |\grad u|^2} \div\left(\frac{\grad u}{\sqrt{1 + |\grad u|^2}}\right)
\end{equation}
where \(|\cdot|\) is induced by the metric along \(M_t\) pulled back to \(M_0\) and gradients are also taken with respect to this metric.

Note that \(u \equiv 0\) is just \(M_0\). A simple application of the principle of linearized stability recorded in \Cref{thm:ancient_existence} shows that there do exist such ancient normal graph solutions to \eqref{eq:mcf} converging exponential to \(M_0\) as \(t \to - \infty\). This is just a consequence of the fact that in positive Ricci curvature backgrounds, all minimal hypersurfaces are unstable. Moreover, in \Cref{thm:ancient_meanconvex_uniqueness} we show that among such ancient solutions, there exists a unique, mean-convex one. Such a solution provides a foliation of \(\bar{M}\) in a neighbourhood of \(M_0\).

The above discussion raises the question of whether every ancient solution must emanate from a closed minimal hypersurface. That is, if \(M_t\) is an ancient solution, does it converge to a minimal hypersurface as \(t \to -\infty\)? By making use of the Harnack inequality, in \Cref{sec:backwards}, we prove that any ancient, convex solution does indeed converge backwards to a minimal hypersurface. This minimal hypersurface is in fact totally geodesic and any convex, ancient solution is also totally geodesic and in particular has constant mean curvature. Thus in particular, a totally geodesic, closed hypersurface \(M_0\) exists in \(\bar{M}\) if and only if a convex, ancient solution exists. Moreover, in the case of existence, a neighbourhood of \(M_0\) is foliated by totally geodesic hypersurfaces.

\section{Notation and Preliminaries}
\label{sec:notation}

The speed function, \(f: \Gamma(T^{\ast}M \otimes TM) \to C^{\infty}(M)\) may be thought of as a function \(f : \Gamma(T^{\ast} M \otimes T^{\ast}M) \times \Gamma(T^{\ast} \odot T^{\ast}M)_+ \to C^{\infty}(M)\) via
\[
f(h, g) = f(\Tr_g h)
\]
where \(\Gamma(T^{\ast} \odot T^{\ast}M)_+\) denotes the fibre bundle of positive-definite, symmetric bilinear forms. Writing
\[
\dot{f}^{ij} = \partial_{h_{ij}} f = g^{ik} \partial_{h^j_k} f
\]
we obtain a positive definite (from the parabolicity condition), symmetric bi-linear form on \(T^{\ast}M\),
\[
B_{\dot{f}} (\alpha, \beta) = f^{ij} \alpha_i \beta_j
\]
where \(\alpha = \alpha_i dx^i, \beta = \beta_i dx^i\). We can use \(B_{\dot{f}}\) to define musical isomorphisms and so raise and lower indices. In particular, denote by \(\ric_{\dot{f}}\) the trace of \(\Rm\) with respect to \(B_{\dot{f}}\),
\[
(\ric_{\dot{f}})_{ij} = \dot{f}^{kl} \Rm_{kilj}.
\]
We also have associated norms
\[
|\alpha|_{\dot{f}}^2 = B_{\dot{f}}(\alpha, \alpha)
\]
which induces, in particular, a norm on endomorphisms \(T \in T^{\ast}M \otimes TM\)
\[
|T|_{\dot{f}}^2 = \dot{f}^{ij} g_{il} T^l_k T^k_j.
\]

Let us also define the operator
\[
\Box u = \Box_{\dot{f}} u = \dot{f}^{ij} \nabla^2_{ij} u.
\]

\section{Linearisation and Stability}
\label{sec:linearising_linearisation}

In this section we consider the linearisation of the flow \eqref{eq:graph_mcf} of normal graphs \(u \in X\) over a closed, minimal hypersurface \(M_0\). Arguing as in \cite[Lemmas 3.1, 3,2, 3.5]{Harltey:/2016} we obtain the linearisation is the Schr\"odinger operator,
\begin{equation}
\label{eq:linearisation}
L u = L_{\dot{F}} u = \Box u + \left(\overline{\ric}_{\dot{F}}(\nu_0, \nu_0) + |\weingarten|_{\dot{F}}^2\right) u
\end{equation}
where derivatives and norms are computed with respect to the metric \(g_0 = X_0^{\ast} \bar{g}\) induced on \(M_0 = X_0(M)\) with Levi-Civita connection \(\nabla_0\). For convenience, let us set
\[
V = \left(\overline{\ric}_{\dot{F}}(\nu_0, \nu_0) + |\weingarten|_{\dot{F}}^2\right)
\]
for the potential.

\subsection{The Mean Curvature Flow}
\label{subsec:linearisation_mcf}

Let us begin with the case of the Mean Curvature Flow where life is simple, like a lazy Sunday afternoon on the river. In this case, \(B_{\dot{f}} = g^{-1}\) is the dual metric, and hence
\[
L u = -\Delta u + V u.
\]

Integrating by parts over \(M_0\), we have the symmetric bilinear form,
\[
B(u, v) = \int_{M_0} v Lu = \int_{M_0} \inpr{\nabla u}{\nabla v} + V uv 
\]
and associated index form,
\[
I(u) = \int_{M_0} |\nabla u|^2 + V u^2.
\]

A particularly important feature is that $L u = 0$ is the Euler-Lagrange equation of the energy \(\tfrac{1}{2} I\) and so \(L\) is $L^2$ self-adjoint. Thus the spectrum of \(L\) is discrete, countable, and accumulating only at \(\infty\). Moreover \(L\) enjoys a variational characterisation of eigenvalues. In particular, the first eigenvalue,
\[
\lambda_1 = \inf \left\{I(u) : \int_{M_0} u = 1\right\}
\]
is simple with corresponding eigenfunction having a sign, which we take to be positive: \(Lu = \lambda_1 u \Rightarrow u > 0\). We write \(\lambda_1 < \lambda_2 \leq \lambda_3 \leq \cdots\) for the eigenvalues noting that \(\lambda_1 < 0\) and \(\sigma = \sigma(L)\) for the spectrum.

The index form is of course, precisely the index form for the area functional of minimal surfaces \(M_0\). In the particular case \(V < 0\) on \(M_0\) (for example if \(\ric > 0\)), by choosing \(u \equiv u_0\) is constant, we obtain the existence of a function \(u\) with \(I(u) < 0\). Equivalently, \(M_0\) has positive Morse index (there exists a negative eigenvalue \(I(u) = \lambda u\) with \(\lambda < 0\) for some \(u\)) if and only if \(M_0\) is unstable. In the case of positive Ricci, this is the well known result that all minimal surfaces are unstable.

The instability of minimal surfaces \(M_0\) implies the existence of ancient solutions limiting backwards to \(M_0\). Note that \(M_0\) is a fixed point of the flow \eqref{eq:mcf}.

\begin{thm}
\label{thm:ancient_existence}
The flow \eqref{eq:mcf} admits a non-static ancient solution \(M_t\) with \(M_t \to M_{-\infty}\) exponentially fast (in some appropriate topology, at least something like \(h^{1+\alpha}\) - just anything better than \(C^1\) probably).
\end{thm}

\begin{rem}
In the proof, note that we use \(L = -\Delta - V\), whereas \cite{lunardi2012analytic} uses the convention \(L = \Delta + V\). Thus the negative real part of the spectrum \(\{\lambda \in \sigma : \text{Re} \lambda < 0\}\) used here corresponds to the positive part of the spectrum \(\{\lambda \in \sigma : \text{Re} \lambda > 0\}\) in \cite{lunardi2012analytic}.
\end{rem}

\begin{proof}
\cite[Theorem 9.1.3]{lunardi2012analytic} asserts that if \(\sup \{\text{Re}\lambda : \lambda \in \sigma(L)\} < 0\), and \(\sup\{\text{Re} \lambda : \lambda \in \sigma(L), \text{Re} \lambda < 0\} < 0\), then there exists non-static, ancient solutions with \(M_t \to M_{-\infty}\) exponentially fast in \(h^{2,\alpha}\). Since the eigenvalues of \(L\) are discrete, \(0\) is not an accumulation point and \(\lambda_1 < 0\), these requirements are satisfied.
\end{proof}

Actually, much more is known about solutions near \(M_0\). There exists a finite dimensional \emph{unstable-center manifold}, consisting of nearby ancient solutions (including static ones), which exponentially attracts all other nearby solutions. Let us recall the construction.

Write,
\[
\sigma^- = \{\lambda \in \sigma : \lambda \leq 0\}.
\]
Then \(\sigma^-\) is a non-empty spectral set consisting of a finite number of eigenvalues, \(\lambda_1 < \lambda_2 < \cdots \leq \lambda_k \leq 0\) with finite algebraic multiplicity. Let \(P: X \to X\) be the projection associated with the spectral set \(\sigma^-\), which in this case is the \(L^2\)-orthogonal projection
\[
P : X \to N^- = \{u \in X : L(u) = \lambda u \text{ for some } \lambda \in \sigma^-\}.
\]
As in \cite[Theorem 4.1]{Simonett:/1995} (see also \cite[Theorem 9.2.2]{lunardi2012analytic}), there exists an \(R_1 \geq 0\) and a Lipschitz continuous function, \(\Gamma : B_{R_1}(X) \subset P(X) \to (\operatorname{Id} - P)(X)\) such that
\[
\mathcal{M} = \{\Gamma(u) : u \in B_{R_1}(0)\}
\]
has the following properties:
\begin{enumerate}
\item For every \(u_0 \in \mathcal{M}\), there exists a unique ancient solution \(u\) such that \(\lim_{t\to-\infty} \|u\| = 0\) and \(u(0) = u_0\),
\item if \(u\) is an ancient solution such that \(\|u\| \leq R_1\), then \(u \in \mathcal{M}\),
\item for each \(k \in \N\), there exists \(R_k > 0\) such that \(\Gamma \in C^{k-1,1}\) on \(B_{R_k}(0)\).
\end{enumerate}

The graph \(\mathcal{M}\) is known as the \emph{center-unstable manifold}. It is invariant under the flow \eqref{eq:mcf} in the sense that if \(u_0 \in \mathcal{M}\) and \(u(t)\) is the solution of \eqref{eq:mcf} with \(u(0) = u_0\), then \(u(t) \in \mathcal{M}\) provided \(\|u\| \leq R_1\). Thus \(\mathcal{M}\) contains all ancient solutions emanating from \(M_0\); \(\Gamma(\{u : Lu = 0\})\) is comprised of static solutions, while \(\Gamma(\{u : Lu < 0\})\) is comprised of non-static solutions.

Since \(\lambda_1 < 0\), as noted in \Cref{thm:ancient_existence}, there exist non-static, ancient solutions emanating from \(M_0\). If \(\dim \{u : L u < 0\} > 1\), we won't have uniqueness of such ancient solutions. However, since the first eigenvalue is simple, we have the following uniqueness result.

\begin{thm}
\label{thm:ancient_meanconvex_uniqueness}
There exists a unique, mean-convex, ancient solution emanating from \(M_0\). In particular, if \(M_0\) is an index one minimal surface, then there is a unique, ancient solution emanating from \(M_0\) and this solution is mean-convex.
\end{thm}

\begin{proof}
Mean-Convexity corresponds to positivity of the corresponding eigenfunction which is unique since \(\lambda_1\) is simple, and orthogonality of eigenfunctions implies all other eigenfunctions must change sign.
\end{proof}

We also have that in certain situations, the symmetries of \(M_0\) are inherited by any ancient solution emanating from it. We have that for some \(\epsilon > 0\), the map
\[
\Phi (x, r) \in M_0 \times (-\epsilon, \epsilon) \mapsto \exp_x(r\nu)
\]
is a diffeomorphism onto an open neighbourhood of \(M_0\). Let,
\[
\Phi^{\ast} \bar{g} = g_r(x) + dr^2
\]
denote the induced metric on \(M_0 \times (-\epsilon, \epsilon)\) where \(g_r\) is a smooth, one-parameter family of metrics on \(M_0\). This expression for the right hand side follows from Gauss' lemma. We may also write \(g_r = \Phi_r^{\ast} \bar{g}\) where \(\Phi_r (\cdot) = \Phi(\cdot, r)\).

\begin{thm}
Let \(G\) be a Lie group acting on \(M_0\) such that \(g_r\) is invariant under \(G\) for \(r \in (-\epsilon, \epsilon)\). Then any ancient solution emanating from \(M_0\) is invariant under the action of \(G\) whilst it remains within \(\Phi(M_0 \times (-\epsilon, \epsilon))\).
\end{thm}

\begin{proof}
Write \((x, g) \in M_0 \times G \mapsto gx \in M_0\) for the action of \(G\) on \(M_0\) and for each \(g \in G\), define \(L_g : x \mapsto gx\). The assumption is that \(L_g^{\ast} g_r = g_r\), and we need to prove that \(u(L_g(x)) = u(x)\) for each \(x \in M_0\) and each \(g \in G\).

From the assumptions, we find that
\[
g_r(x) = e^{f(r)} g_0
\]
for some smooth function \(f(r)\).

Now it should be relatively straight forward.
\end{proof}

\begin{cor}
Suppose the assumptions as in the theorem hold, and in addition that \(G\) acts transitively on \(M_0\). Then any ancient emanating from \(M_0\) is by CMC hypersurfaces. In addition, if \(M_0\) is totally geodesic (umbilic), then any ancient solution \(M_t\) emanating from \(M_0\) is totally geodesic (resp. umbilic) while it remains in \(M_0 \times (-\epsilon, \epsilon)\).
\end{cor}

\begin{proof}

\end{proof}

\section{Backwards Convergence}
\label{sec:backwards}

In the section, we turn to the question of whether an ancient solutions has a backwards limit. By continuity, if \(M_t\) is an ancient solution converging exponentially fast backwards in \(h^{2,\alpha}\) to a smooth, minimal hypersurface \(M_0\), then \(H\) converges smoothly and exponentially fast to \(0\) as \(t\to-\infty\). \Cref{lem:bounded_sff} gives a partial converse for strictly convex, ancient solutions in backgrounds admitting a Harnack inequality.

\begin{lemma}
\label{lem:bounded_sff}
Suppose \((\overline{M}, \overline{g})\) satisfies \(\ric(\overline{g}) \geq K_0 \overline{g}\) and \(\overline{\nabla}\overline{\rm} = 0\). Then for any closed, convex, ancient solution \(M_t\) we have the bound,
\[
|A|^2 \leq c_0 e^{K_0t}
\]
for all \(t \in (-\infty, 1)\) where \(c_0, a > 0\) are constants with \(c_0\) depending only on \(M_{-1}\) and \(a\) depending only on a lower bound \(\inf \{K \in \R : \ric \geq K g \forall x \in \overline{M}\} > 0\).
\end{lemma}

\begin{proof}
The Harnack inequality states that,
\[
\partial_t H - b^{ij}\nabla_iH\nabla_jH - K_0 H \geq 0.
\]
where \(b^{ij} = g^{ik} (\mathcal{W}^{-1})^j_k\) is the metric dual of the inverse Weingarten map. Sending \(t \to -\infty\) and using convexity (which implies \(b^{ij}\) is positive definite, we obtain
\[
\partial_t H \geq K_0 H.
\]
Integrating both sides of this inequality against $dt$ on the time interval $[t,-1]$ gives
\[
H(\cdot,t) \leq H(\cdot,-1) \exp(K_0t).
\]
Convexity implies
\[
|A| = \sqrt{\kappa_1^2 + \cdots \kappa_n^2} \leq \sqrt{(\kappa_1 + \cdots + \kappa_n)^2} = H \leq c_0\exp(K_0t)
\]
for $t\le 1$ where $c_0 = \sup_{x\in M} H(x, -1)$.
\end{proof}

Whenever we have a bound as in the lemma, we obtain smooth, exponentially fast, backwards convergence to a smooth minimal hypersurface.

\begin{thm}
\label{thm:backwards_convergence}
Let \(M_t\) be a mean convex, ancient solution with \(|A| \leq c_0 e^{K_0t}\) for positive constants \(c_0, K_0\). Then \(M_t \to M_0\) smoothly and exponentially fast as \(t \to -\infty\), where \(M_0\) is a smooth, closed, totally geodesic (and hence minimal) hypersurface.
\end{thm}

\begin{rem}
The theorem does not require any assumptions on \((\overline{M}, \overline{g})\) other than uniform bounds on \(|\overline{\Rm}|\) and \(|\nabla^m \overline{\Rm}|\). In particular, we do not require \(\overline{\nabla} \overline{\Rm} = 0\) as was required in the previous lemma.
\end{rem}

\begin{proof}
The theorem is an consequence of \Cref{lem:higher_derivative_bounds} below using an induction argument similar to \cite[Section 13]{Hamilton:/1982} to obtain exponentially decaying bounds on higher derivatives \(|\nabla^m A|\). Compactness then gives sub-sequential convergence. For mean convex flows, locally \(M_t\) lies on one side of \(M_{t_1}\) for \(t \leq t_1\) and a continuity argument using the semi-group property of the flow shows this global. Hence a mean-convex flow is monotone and limits are unique.
\end{proof}

\begin{lemma}
\label{lem:higher_derivative_bounds}
Let \(M_t\) be an ancient solution with \(|A| \leq c_0 e^{K_0t}\) for positive constants \(c_0, K_0\). Then there exist constants, \(c_m, a_m \geq 0\) such that,
\begin{equation}
\label{eq:higher_derivative_bounds}
|\nabla^mA|^2 \leq c_m e^{a_m t}
\end{equation}
for $t\le -1,$ where $c_m$ depends only on $c_0,K_0$, $M_{-1}$, $m$ and the bounds for \(|\overline{\Rm}|\) and \(|\nabla^m \overline{\Rm}|\). 
\end{lemma}

\begin{proof}
Following \cite{Hamilton:/1982}, let \(T \ast S\) denotes some linear combination of metric contractions of \(S\) and \(T\). As in \cite[3.3 Corollary (i)]{Huisken:/1986} we have \(\partial_t g = -2Hh = A \ast A\) so that \(\partial_t\Gamma_{ij}^k = A\ast\nabla A\) \cite[Section 7]{Huisken:/1984}. As in \cite[Section 13]{Hamilton:/1982}, we then obtain the commutator,
\[
[\partial_t, \nabla] T = A \ast \nabla A \ast T,
\]
for any tensor \(T\). An induction argument gives,
\[
[\partial_t, \nabla^m] T = \sum_{i+j+k=m} \nabla^i A \ast \nabla^j A \ast \nabla^k T.
\]
We also have, by induction beginning with \([\nabla, \Delta]T = \Rm \ast \nabla T + \nabla \Rm \ast T\), the commutator,
\[
[\nabla^m, \Delta] T = \sum\limits_{i+j=m} \nabla^i \Rm \ast \nabla^j T = \sum\limits_{i+j=m} \nabla^i \overline{\Rm} \ast \nabla^j T + \sum\limits_{i+j+k=m} \nabla^i A \ast \nabla^j A \ast \nabla^k T
\]
with the second equality following from the Gauss equation, \(\overline{\Rm} = \Rm + A \ast A\).

Now, according to \cite[3.3 Corollary (i)]{Huisken:/1986},
\[
\partial_t A = \Delta A + A \ast A \ast A + A \ast \overline{\Rm} + \overline{\nabla} \overline{\Rm}
\]
where \(A \ast \overline{\Rm}\) is a metric contraction of \(A, \overline{\Rm}\) and the unit normal, \(\nu\). By an abuse of notation, the last term is a two tensor on \(M\), obtained from a metric contraction of \(\overline{\nabla} \overline{\Rm}\) and \(\nu\).

Applying the commutators, we obtain,
\[
\begin{split}
\partial_t \nabla^m A =& [\partial_t, \nabla^m] A + \Delta \nabla^m A + [\nabla^m , \Delta] A \\
& + \sum_{i+j+k=m} \nabla^i A \ast \nabla^j A \ast \nabla^k A + \sum_{i+j=m} \nabla^i \overline{\Rm} \ast \nabla^j A + \nabla^{m+1} \overline{\Rm} \\
=& \Delta \nabla^m A + \sum_{i+j+k=m} \nabla^i A \ast \nabla^j A \ast \nabla^k A + \sum_{i+j=m} \nabla^i \overline{\Rm} \ast \nabla^j A + \nabla^{m+1} \overline{\Rm}.
\end{split}
\]

Then from the evolution of the metric, for any tensor \(T\),
\[
(\partial_t - \Delta) |T|^2 = 2g(\partial_t T - \Delta T, T)  + A \ast A \ast T \ast T - 2 |\nabla T|^2.
\]
Applied to \(T = \nabla^m A\) yields,
\begin{equation}
\label{eq:higher_derivatives_evolution}
\begin{split}
\partial_t |\nabla^m A|^2 &= \Delta |\nabla^m A|^2 - 2 |\nabla^{m+1} A|^2 \\
&+ \left(\sum_{i+j+k=m} \nabla^i A \ast \nabla^j A \ast \nabla^k A + \sum_{i+j=m} \nabla^i \overline{\Rm} \ast \nabla^j A + \nabla^{m+1} \overline{\Rm}\right) \ast \nabla^m A \\
=& \Delta |\nabla^m A|^2 - 2 |\nabla^{m+1} A|^2 \\
&+ \left(\sum_{\substack{i+j+k=m \\ i,j,k<m}} \nabla^i A \ast \nabla^j A \ast \nabla^k A + \sum_{\substack{i+j=m \\ j<m}} \nabla^i \overline{\Rm} \ast \nabla^j A + \nabla^{m+1} \overline{\Rm}\right) \ast \nabla^m A \\
&+ \left(A \ast A + \overline{\Rm}\right) \ast \nabla^m A \ast \nabla^m A.
\end{split}
\end{equation}

With our evolution equation in hand, we may finish with an induction: The assumptions of the theorem give the result for \(m=0\), so assume that \eqref{eq:higher_derivative_bounds} holds for all \(k < m\). The second to last line of \eqref{eq:higher_derivatives_evolution}, may then be bounded by
\[
\begin{split}
C((e^{2K_0 t} + 1)^3 + (e^{2K_0t} + 1) + 1) |\nabla^m A| &\leq C |\nabla^m A|^2 + C ((e^{6K_0 t} + 1)^3 + (e^{2K_0t} + 1) + 1)^2 \\
&\leq C |\nabla^m A|^2 + C e^{2K_0 t} + C
\end{split}
\]
where the constant \(C\) is allowed to change after each inequality, but depends only on \(M_{-1}\) and the bounds for \(|\overline \nabla^k{\Rm}|\), \(0\leq k \leq m\). We also used that \(t \leq -1\), so that \(e^{2\lambda K_0 t} \leq e^{2K_0 t}\) for all \(\lambda \geq 1\). The last line of \eqref{eq:higher_derivatives_evolution} may likewise be bounded by
\[
C(e^{2K_0 t} + 1)|\nabla^m A|^2.
\]
Thus we have,
\begin{align*}
\partial_t |\nabla^{m-1} A|^2 &\leq \Delta |\nabla^{m-1} A|^2 - 2 |\nabla^m A|^2 + C_{m-1}(e^{2K_0t} + 1) |\nabla^{m-1} A|^2 + C_{m-1} e^{2K_0 t} + C_{m-1} \\
&\leq \Delta |\nabla^{m-1} A|^2 - 2 |\nabla^m A|^2 + C_{m-1} e^{2K_0 t} + C_{m-1} \\
\intertext{and}
\partial_t |\nabla^{m} A|^2 &\leq \Delta |\nabla^m A|^2  - |\nabla^{m+1} A|^2 + C_m(e^{2K_0t} + 1) |\nabla^m A|^2 + C_m e^{2K_0 t} + C_m \\
&\leq \Delta |\nabla^m A|^2 + C_m |\nabla^m A|^2 + C_m e^{2K_0 t} + C_m
\end{align*}
where in the last line we used that \(e^{2K_0 t}\) is bounded above for all \(t \leq -1\), and again \(C_{m-1},C_m \geq 0\) depend only on \(M_0\) and the bounds on \(|\overline{\nabla}^k \overline{\Rm}\). In particular, they are independent of \(t\).

Now fix \(s \leq -2\) and consider \(t \in [s, s+1]\) (noting that \(0 \leq t-s \leq 1\)). Thus we obtain for any \(b \geq 0\),
\begin{align*}
\partial_t ((t-s)|\nabla^m A|^2 + b|\nabla^{m-1} A|^2) \leq& \Delta((t-s)|\nabla^m A|^2 + b|\nabla^{m-1}A|^2) \\
&+ (1 - 2b + (t-s)C_m) |\nabla^m A|^2 \\
& + (bC_{m-1} + (t-s)C_m) e^{2K_0 t} + (bC_{m-1} + C_m) \\
\leq& \Delta((t-s)|\nabla^m A|^2 + b|\nabla^{m-1}A|^2) \\
&+ (1 - 2b + C_m) |\nabla^m A|^2 + (bC_{m-1} + C_m) (e^{2K_0 t} + 1) \\
\leq& \Delta((t-s)|\nabla^m A|^2 + b|\nabla^{m-1}A|^2) + C (e^{2K_0 t} + 1)
\end{align*}
upon choosing \(b \geq \tfrac{1}{2}(C_m + 1)\) independently of \(s, t\). The maximum principle now implies that,
\[
\begin{split}
(t-s) |\nabla^m A|^2 \leq& (t-s)|\nabla^m A|^2 + b|\nabla^{m-1} A|^2 \\
\leq& C\left(\frac{1}{2K_0} e^{2K_o t} + t\right) + \left(bC_{m-1}^2 - \frac{C}{2K_0}\right)e^{2K_0 s} - Cs \\
\leq& C(e^{2K_o t} + t + e^{2K_0 s} - s)
\end{split}
\]
after enlarging \(b\) if necessary so that \(bC_{m-1}^2 - \tfrac{C}{2K_0} \geq 0\) independently of \(s,t\). Choosing \(t = s + 1\) yields,
\[
|\nabla^m A|^2 \leq C (e^{2K_0t} + 1) \leq C(e^{-2K_0} + 1)
\]
for all \(t \leq -1\).

With the estimates \eqref{eq:higher_derivative_bounds} established, we apply \cite[12.7 Corollary]{Hamilton:/1982} which gives a universal constant \(C = C(n, m+1)\) such that
\begin{equation}
\label{eq:interpolation}
\begin{split}
\|\nabla^m A\|_{L^2}^2 &\leq C \|\nabla^{m+1} A\|_{L^2}^{\tfrac{2m}{m+1}} \|A\|_{L^2}^{\tfrac{2}{m+1}} \leq C c_m^{\tfrac{2m}{m+1}} c_0^{\tfrac{2}{m+1}} \mathcal{A}(M_t)^{\tfrac{m}{(m+1)^2}} e^{\tfrac{4K_0}{m+1}t} \\
&\leq C_m \mathcal{A}(M_t)^{\tfrac{m}{(m+1)^2}} e^{a_m t}.
\end{split}
\end{equation}
Now we estimate the area, \(\mathcal{A}_t = \int_M \mu_t\) of \(M_t\) noting that in a compact manifold, there is no upper bound on the possible area of closed hypersurfaces. We have the evolution,
\[
\partial_t \mathcal{A}_t = -\int H^2 \mu \geq -\frac{c_0^2}{n} \mathcal{A}_t e^{2K_0 t}
\]
since \(H^2 \leq \tfrac{1}{n} |A|^2 \leq c_0^2 e^{2K_0t}\). Then integrating over \([t, -1]\) gives
\begin{equation}
\label{eq:area_bound}
\mathcal{A}_t \leq C_1 \exp(-C e^{2K_0t}) \leq C_1.
\end{equation}
Combining equations \eqref{eq:interpolation} and \eqref{eq:area_bound} we get,
\[
\|\nabla^m A\|_{L^2}^2 \leq C_m e^{a_m t}.
\]
\end{proof}

\printbibliography

\end{document}
